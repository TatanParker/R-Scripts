\documentclass[]{article}
\usepackage{lmodern}
\usepackage{amssymb,amsmath}
\usepackage{ifxetex,ifluatex}
\usepackage{fixltx2e} % provides \textsubscript
\ifnum 0\ifxetex 1\fi\ifluatex 1\fi=0 % if pdftex
  \usepackage[T1]{fontenc}
  \usepackage[utf8]{inputenc}
\else % if luatex or xelatex
  \ifxetex
    \usepackage{mathspec}
  \else
    \usepackage{fontspec}
  \fi
  \defaultfontfeatures{Ligatures=TeX,Scale=MatchLowercase}
\fi
% use upquote if available, for straight quotes in verbatim environments
\IfFileExists{upquote.sty}{\usepackage{upquote}}{}
% use microtype if available
\IfFileExists{microtype.sty}{%
\usepackage{microtype}
\UseMicrotypeSet[protrusion]{basicmath} % disable protrusion for tt fonts
}{}
\usepackage[margin=1in]{geometry}
\usepackage{hyperref}
\hypersetup{unicode=true,
            pdftitle={TAREA M3},
            pdfauthor={TATAN RUFINO},
            pdfborder={0 0 0},
            breaklinks=true}
\urlstyle{same}  % don't use monospace font for urls
\usepackage{graphicx,grffile}
\makeatletter
\def\maxwidth{\ifdim\Gin@nat@width>\linewidth\linewidth\else\Gin@nat@width\fi}
\def\maxheight{\ifdim\Gin@nat@height>\textheight\textheight\else\Gin@nat@height\fi}
\makeatother
% Scale images if necessary, so that they will not overflow the page
% margins by default, and it is still possible to overwrite the defaults
% using explicit options in \includegraphics[width, height, ...]{}
\setkeys{Gin}{width=\maxwidth,height=\maxheight,keepaspectratio}
\IfFileExists{parskip.sty}{%
\usepackage{parskip}
}{% else
\setlength{\parindent}{0pt}
\setlength{\parskip}{6pt plus 2pt minus 1pt}
}
\setlength{\emergencystretch}{3em}  % prevent overfull lines
\providecommand{\tightlist}{%
  \setlength{\itemsep}{0pt}\setlength{\parskip}{0pt}}
\setcounter{secnumdepth}{0}
% Redefines (sub)paragraphs to behave more like sections
\ifx\paragraph\undefined\else
\let\oldparagraph\paragraph
\renewcommand{\paragraph}[1]{\oldparagraph{#1}\mbox{}}
\fi
\ifx\subparagraph\undefined\else
\let\oldsubparagraph\subparagraph
\renewcommand{\subparagraph}[1]{\oldsubparagraph{#1}\mbox{}}
\fi

%%% Use protect on footnotes to avoid problems with footnotes in titles
\let\rmarkdownfootnote\footnote%
\def\footnote{\protect\rmarkdownfootnote}

%%% Change title format to be more compact
\usepackage{titling}

% Create subtitle command for use in maketitle
\providecommand{\subtitle}[1]{
  \posttitle{
    \begin{center}\large#1\end{center}
    }
}

\setlength{\droptitle}{-2em}

  \title{TAREA M3}
    \pretitle{\vspace{\droptitle}\centering\huge}
  \posttitle{\par}
    \author{TATAN RUFINO}
    \preauthor{\centering\large\emph}
  \postauthor{\par}
      \predate{\centering\large\emph}
  \postdate{\par}
    \date{10 de mayo de 2019}


\begin{document}
\maketitle

\begin{enumerate}
\def\labelenumi{\arabic{enumi}.}
\tightlist
\item
  INTRODUCCIÓN
\end{enumerate}

Con este proyecto trataré de hacer un modesto análisis de datos sobre la
base de datos student proporcionada por la comunidad internacional y
sobre la que se centra la tarea en la que nos encontramos. Me he ayudado
de la comunidad internacional, que ofrece todo tipo de ayuda en análisis
para bases de datos conocidas, razón por la cual usaré funciones y
metologías no mostradas hasta el momento

Este proyecto tiene como objetivo descubrir la influencia de diversas
variables y parámetros en las notas de un estudiante de secundaria. En
este caso nos centraremos en el estudiante de matemáticas, las
observaciones fueron recabadas haciendo la encuesta de un total de 395
estudiantes, que viene dado por el dataset studentMat.csv.

El dataset cuenta con 33 variables diferentes, el cual será sometido a
la siguiente metodología de acciones: - INICIO: Declaración de
librerías, importación de tablas, primeras visualizaciones - Filtrado de
dataset y definición de variables clave - Análisis exploratorio y
visualizaciones - Comentarios - Aplicación de algoritmos de Machine
learning - Algoritmo de Arbol de decisiones - Algoritmo de Regresión
simple - Evaluación

\begin{enumerate}
\def\labelenumi{\arabic{enumi}.}
\setcounter{enumi}{1}
\tightlist
\item
  INICIO
\end{enumerate}

temp \textless{}- tempfile()
download.file(``\url{http://archive.ics.uci.edu/ml/machine-learning-databases/00356/student.zip}'',temp,
mode=``wb'') unzip(temp, ``student-mat.csv'') ST \textless{}-
read.table(``student-mat.csv'',sep= ``;'', header= T) unlink(temp)

ST{[},1:5{]} \%\textgreater{}\% head() \%\textgreater{}\% kable()

summary(head(ST{[},1:5{]})

\begin{enumerate}
\def\labelenumi{\arabic{enumi}.}
\setcounter{enumi}{2}
\tightlist
\item
  FILTRADO Y DEFINICION DE VARIABLES CLAVE
\end{enumerate}

temp \textless{}- tempfile()
download.file(``\url{http://archive.ics.uci.edu/ml/machine-learning-databases/00356/student.zip}'',temp,
mode=``wb'') unzip(temp, ``student-mat.csv'') ST \textless{}-
read.table(``student-mat.csv'',sep= ``;'', header= T) unlink(temp)

\begin{verbatim}

#Primeras visualizaciones
\end{verbatim}

data{[},1:5{]} \%\textgreater{}\% head() \%\textgreater{}\% kable()
summary(ST) names(ST)

\section{Compruebo si hay NA}\label{compruebo-si-hay-na}

ST \%\textgreater{}\% is.na() \%\textgreater{}\% all()

\section{Consulto la codificacion de la DT de la pagina de
google}\label{consulto-la-codificacion-de-la-dt-de-la-pagina-de-google}

COL\_LIST \textless{}-
gsheet2tbl(`\url{https://docs.google.com/spreadsheets/d/1mDsF0aMNgODx7063l2mxV1_zP32fAe_P55SBmtG72G8}')
kable(COL\_LIST) kable(COL\_LIST{[},1:3{]})

\section{Escogemos 18 variables como
predictores}\label{escogemos-18-variables-como-predictores}

\subsubsection{7 variables categóricas}\label{variables-categoricas}

\subsubsection{11 variables numericas}\label{variables-numericas}

\section{Y finalmente 3 variables dependientes y susceptibles de
predicción, correspondientes a las
notas}\label{y-finalmente-3-variables-dependientes-y-susceptibles-de-prediccion-correspondientes-a-las-notas}

\section{\texorpdfstring{En base a lo que observo en nuestro ST, y dadas
las características de las variables actuales, eliminaré diversas
variables entre las que se encuentran ``famsize'' ``reason'' o
``guardian'' o ``school'', ya que serán irrelevantes en nuestro análisis
con
select()}{En base a lo que observo en nuestro ST, y dadas las características de las variables actuales, eliminaré diversas variables entre las que se encuentran famsize reason o guardian o school, ya que serán irrelevantes en nuestro análisis con select()}}\label{en-base-a-lo-que-observo-en-nuestro-st-y-dadas-las-caracteristicas-de-las-variables-actuales-eliminare-diversas-variables-entre-las-que-se-encuentran-famsize-reason-o-guardian-o-school-ya-que-seran-irrelevantes-en-nuestro-analisis-con-select}

ST \textless{}- ST
\%\textgreater{}\%as\_tibble()\%\textgreater{}\%select(sex, age,
address,Pstatus, Medu, Fedu, Mjob,
Fjob,studytime,traveltime,failures,higher,internet, goout,
Dalc,Walc,health, absences,G1,G2,G3) dim(ST)

ST \%\textgreater{}\% glimpse() ST \%\textgreater{}\% summary()

\begin{enumerate}
\def\labelenumi{\arabic{enumi}.}
\setcounter{enumi}{3}
\tightlist
\item
  ANÁLISIS EXPLORATORIO DE DATOS DIVERSO
\end{enumerate}

\section{ANALISIS EXPLORATORIO DE
DATOS}\label{analisis-exploratorio-de-datos}

\section{Analizaremos los siguientes
indicadores}\label{analizaremos-los-siguientes-indicadores}

\section{1. Según género}\label{segun-genero}

\section{2. Consumo de alcohol (W/D)}\label{consumo-de-alcohol-wd}

\section{3. Costumbres, objetivos y origen del
alumno}\label{costumbres-objetivos-y-origen-del-alumno}

\section{4. Salud y asistencia a
clase}\label{salud-y-asistencia-a-clase}

\section{5. Acceso a internet}\label{acceso-a-internet}

\section{6. Relación asistencia a clase con las
notas}\label{relacion-asistencia-a-clase-con-las-notas}

\subsubsection{Según Genero}\label{segun-genero}

(ANA\_1\textless{}-ST\%\textgreater{}\%
mutate(pass=ifelse(G3\textgreater{}=10,1,0), fail=
ifelse(G3\textless{}10,1,0))\%\textgreater{}\%
filter(sex==``F''\textbar{}sex==``M'')\%\textgreater{}\%
group\_by(sex)\%\textgreater{}\% summarise(Pass=sum(pass),
Fail=sum(fail)))

ANA\_1\%\textgreater{}\% ggplot(aes(x=sex,y=Fail))+
geom\_bar(stat=``identity'') ANA\_1

\subsubsection{Observamos que hay diferencias notables entre mujeres y
hombres}\label{observamos-que-hay-diferencias-notables-entre-mujeres-y-hombres}

\section{2. Consumo de alcohol}\label{consumo-de-alcohol}

ANA\_2a \textless{}- ST\%\textgreater{}\%
group\_by(Walc)\%\textgreater{}\% aggregate(G3\textasciitilde{}Walc,
data=., mean)\%\textgreater{}\% arrange(desc(G3)) ANA\_2a

ANA\_2b \textless{}- ST\%\textgreater{}\%
group\_by(Dalc)\%\textgreater{}\% aggregate(G3\textasciitilde{}Dalc,
data=., mean)\%\textgreater{}\% arrange(desc(G3)) ANA\_2b

\section{Vemos que disminuye el promedio a medida que aumenta el consumo
de alcohol entre semana, pero no hay un patron para fin de
semana}\label{vemos-que-disminuye-el-promedio-a-medida-que-aumenta-el-consumo-de-alcohol-entre-semana-pero-no-hay-un-patron-para-fin-de-semana}

\section{Cruce de datos para visualizacion de aspectos que relacionen
datos relacionados con el género y el consumo de
alcohol}\label{cruce-de-datos-para-visualizacion-de-aspectos-que-relacionen-datos-relacionados-con-el-genero-y-el-consumo-de-alcohol}

DUM \textless{}- dummyVars(``\textasciitilde{}.'', data=ST) GEN\_ALC
\textless{}- data.frame(predict(DUM, newdata=ST)) correl1
\textless{}-cor(GEN\_ALC{[},c(``G3'',``sex.F'',``sex.M'',``Walc'',``Dalc''){]})
source(``\url{https://raw.githubusercontent.com/briatte/ggcorr/master/ggcorr.R}'')
correl1 \%\textgreater{}\% ggcorr(label = TRUE)+ ggtitle(``Correlaciones
entre genero, consumo de alcohol y las notas'')

\section{Pasamos}\label{pasamos}

ST\(Dalc <- as.factor(ST\)Dalc) ST\(Walc <- as.factor(ST\)Walc)
P1a\textless{}-ST \%\textgreater{}\% ggplot(aes(x=Dalc, y=G3, fill=
Dalc))+ geom\_boxplot()+ coord\_flip()+ xlab(``Consumo de alcohol entre
semana'')+ ylab(``Notas'')+ facet\_grid(\textasciitilde{}sex)
P1b\textless{}-ST \%\textgreater{}\% ggplot(aes(x=Walc, y=G3, fill=
Walc))+ geom\_boxplot()+ coord\_flip()+ xlab(``Consumo de alcohol
entresemana'')+ ylab(``Notas'')+ facet\_grid(\textasciitilde{}sex)
grid.arrange(P1a,P1b,ncol=2)

\section{Básicamente podemos predecir que el consumo de alcohol tiene un
impacto mucho mayor que el género en las
notas}\label{basicamente-podemos-predecir-que-el-consumo-de-alcohol-tiene-un-impacto-mucho-mayor-que-el-genero-en-las-notas}

\section{3. Costumbres, objetivos y origen del
alumno}\label{costumbres-objetivos-y-origen-del-alumno-1}

class(ST\(goout) ST\)goout \textless{}- as.factor(ST\$goout)
\#Convertimos a factor para que me acepte el algoritmo ANA\_3
\textless{}- ST\%\textgreater{}\% group\_by(goout)\%\textgreater{}\%
summarise(Avr= mean(G3,na.rm=TRUE))\%\textgreater{}\% arrange(desc(Avr))
ANA\_3

\section{Vemos que hay un decrecimiento en las notas en cuanto que el
alumno supera el factor de salida de
4}\label{vemos-que-hay-un-decrecimiento-en-las-notas-en-cuanto-que-el-alumno-supera-el-factor-de-salida-de-4}

P2a\textless{}-ST \%\textgreater{}\%
group\_by(address)\%\textgreater{}\% ggplot(aes(x=factor(Dalc), y= G3))+
geom\_jitter(alpha=0.6)+ scale\_x\_discrete(``Alcohol entresemana'')+
scale\_y\_continuous(``Notas'')+ facet\_grid(\textasciitilde{}address)
P2b\textless{}-ST \%\textgreater{}\%
group\_by(address)\%\textgreater{}\% ggplot(aes(x=factor(Walc), y= G3))+
geom\_jitter(alpha=0.6)+ scale\_x\_discrete(``Alcohol en fin de
semana'')+ scale\_y\_continuous(``Notas'')+
facet\_grid(\textasciitilde{}address) grid.arrange(P2a,P2b,ncol=2)

\section{Otro ejemplo paradigmático de la relacion del decrecimiento de
notas respecto al consumo de alcohol. Se observa que se bebe menos en
las zonas
rurales}\label{otro-ejemplo-paradigmatico-de-la-relacion-del-decrecimiento-de-notas-respecto-al-consumo-de-alcohol.-se-observa-que-se-bebe-menos-en-las-zonas-rurales}

ST\%\textgreater{}\% ggplot(aes(x=higher, y=G3))+ geom\_boxplot()+
facet\_grid(\textasciitilde{}sex)

\section{Los alumnos que aspiran a mejor educacion, tienen mejores
notas. Los hombres son mejores que las mujeres
también}\label{los-alumnos-que-aspiran-a-mejor-educacion-tienen-mejores-notas.-los-hombres-son-mejores-que-las-mujeres-tambien}

\section{4. Salud y asistencia a
clase}\label{salud-y-asistencia-a-clase-1}

ST\%\textgreater{}\% group\_by(sex)\%\textgreater{}\%
ggplot(aes(x=factor(health), y=absences, color=sex))+
geom\_smooth(aes(group=sex), method=``lm'', se=FALSE)

\section{Este interesante gráfico nos muestra que existe una relación
lineal decreciente entre la salud del alumno y las ausencias en
clase}\label{este-interesante-grafico-nos-muestra-que-existe-una-relacion-lineal-decreciente-entre-la-salud-del-alumno-y-las-ausencias-en-clase}

\section{Ademas de ser, de nuevo, las mujeres quienes faltan más clase,
siendo sus ausencias menos dependientes con la
salud}\label{ademas-de-ser-de-nuevo-las-mujeres-quienes-faltan-mas-clase-siendo-sus-ausencias-menos-dependientes-con-la-salud}

\section{5. Acceso a internet}\label{acceso-a-internet-1}

ST\%\textgreater{}\% group\_by(internet)\%\textgreater{}\%
ggplot(aes(x=G3, fill=internet))+ geom\_density( alpha=0.8)

\section{el uso de internet afecta a las notas, aunque no en
exceso}\label{el-uso-de-internet-afecta-a-las-notas-aunque-no-en-exceso}

\section{6. Relación asistencia a clase con las
notas}\label{relacion-asistencia-a-clase-con-las-notas-1}

P3 \textless{}- ggplot(ST, aes(absences, G3)) P3 + geom\_point() +
geom\_smooth(method=``lm'', se=F) + labs(y=``G3'', x=``abcenses'',
title=``Comparativa ausencias con notas'')

ggplot(ST, aes(x=absences, y=G3)) + geom\_bar(stat=``identity'',
width=.5, fill=``tomato3'') + labs(y=``G3'', x=``absences'',
title=``Ausencias vs Notas'')

\begin{enumerate}
\def\labelenumi{\arabic{enumi}.}
\setcounter{enumi}{4}
\tightlist
\item
  ANALISIS NO SUPERVISADO SEGÚN CLUSTERING CON ALGORITMO KMEANS
\end{enumerate}

CLUSTERING DE DATOS

\subsubsection{CREAMOS ST\_MOD COMO REFERENCIA CON EL G3 REESCALADO DE
1:10}\label{creamos-st_mod-como-referencia-con-el-g3-reescalado-de-110}

ST\_MOD=ST ST\_MOD\(G3=as.integer(ST_MOD\)G3)

ST\_MOD\(G3=rescale(ST_MOD\)G3,to=c(1,10))

\subsubsection{convertimos a integer}\label{convertimos-a-integer}

ST\_KM=
ST{[}c(``age'',``Medu'',``Fedu'',``studytime'',``traveltime'',``failures'',``goout'',``Dalc'',``Walc'',``health'',``absences'',``G1'',``G2''){]}
ST\_factors = ST\_KM \%\textgreater{}\% select\_if(is.factor)
\%\textgreater{}\% colnames() ST\_KM{[},ST\_factors{]} =
lapply(ST\_KM{[},ST\_factors{]}, as.integer)

\#Limpiamos variables de valores dispersos \#abcenses
plot(ST\_KM\(absences)  ST_KM\)absences\textless{}-
floor(rescale(ifelse(ST\_KM\(absences>3*mean(ST_KM\)absences),
3*mean(ST\_KM\(absences),ST_KM\)absences), ,to=c(1,5)))
\#\#\#plot(ST\_KM\(absences)  #age  plot(ST_KM\)age)
ST\_KM\(age<- floor(rescale(ifelse(ST_KM\)age\textgreater{}19,
19,ST\_KM\(age),  ,to=c(1,5)))  plot(ST_KM\)age)

\#\#Se reajusta todo el sistema a escala 1:5

\section{\texorpdfstring{ST\_KM\(absences<- floor(rescale(ST_KM\)absences,to=c(1,5)))}{ST\_KMabsences\textless{}- floor(rescale(ST\_KMabsences,to=c(1,5)))}}\label{st_kmabsences--floorrescalest_kmabsencestoc15}

ST\_KM{[}, c(1:13){]} \textless{}- lapply(ST\_KM{[}, c(1:13){]},
function(x) rescale(x,to=c(1,5))) ST\_KM=floor(ST\_KM) ST\_factors =
ST\_KM \%\textgreater{}\% select\_if(is.factor) \%\textgreater{}\%
colnames() ST\_KM{[},ST\_factors{]} = lapply(ST\_KM{[},ST\_factors{]},
as.integer)

glimpse(ST\_KM) summary(ST\_KM)

mydata \textless{}- ST\_KM wss \textless{}-
(nrow(mydata)-1)*sum(apply(mydata,2,var)) for (i in 2:15) wss{[}i{]}
\textless{}- sum(kmeans(mydata,centers=i)\$withinss)

plot(1:15, wss, type=``b'', xlab=``Numero de Clusters'', ylab=``Sumas de
cuadrados dentro de los grupos'', main=``Num de clusters óptimo según
Elbow'', pch=20, cex=2)

\subsection{se deciden 5 centros}\label{se-deciden-5-centros}

KM=kmeans(ST\_KM,5) KM

table(ST\_MOD\(G3, KM\)cluster)

plot(ST\_KM\(age, col=KM\)cluster)

plot(KM\(centers) radial.plot(KM\)centers{[}1,{]},
labels=names(KM\$centers{[}1,{]}), rp.type=``s'', radial.lim=c(0,8),
point.symbols=13, point.col=``red'', mar = c(2,1,5,2))

plot(ST\_KM \%\textgreater{}\% select(age, absences), col =
KM\(cluster) points(as.data.frame(KM\)centers) \%\textgreater{}\%
select(age, absences), col = 1:3, pch = 8, cex = 2)

ST\_FIN \textless{}- ST\_MOD \%\textgreater{}\% mutate(cluster\_id =
KM\$cluster) kable(head(ST\_FIN))

\section{Aunque es posible que haber normalizado los valores de las
variables entre 1:5 puede que nos haya limitado
la}\label{aunque-es-posible-que-haber-normalizado-los-valores-de-las-variables-entre-15-puede-que-nos-haya-limitado-la}

\section{vision grafica de las dispersiones, en general vemos, tras un
proceso iterativo en la busqueda de
centros}\label{vision-grafica-de-las-dispersiones-en-general-vemos-tras-un-proceso-iterativo-en-la-busqueda-de-centros}

\section{\texorpdfstring{Todo apunta a que aquellas variables como
G1,G2,``studytime'', ``traveltime'' se encuentran en el grueso de
valores máximos asociados a notas
altas}{Todo apunta a que aquellas variables como G1,G2,studytime, traveltime se encuentran en el grueso de valores máximos asociados a notas altas}}\label{todo-apunta-a-que-aquellas-variables-como-g1g2studytime-traveltime-se-encuentran-en-el-grueso-de-valores-maximos-asociados-a-notas-altas}

\section{\texorpdfstring{Y las variables de tipo Walc,Dalc,``goout''
Tienen mayor influencia en el sentido inverso, y están asociados a la
caida de
notas.}{Y las variables de tipo Walc,Dalc,goout Tienen mayor influencia en el sentido inverso, y están asociados a la caida de notas.}}\label{y-las-variables-de-tipo-walcdalcgoout-tienen-mayor-influencia-en-el-sentido-inverso-y-estan-asociados-a-la-caida-de-notas.}

\subsection{\texorpdfstring{Mis fuentes principales son el
`table(ST\_MOD\(G3, KM\)cluster)' y el esquema de centros
radial.}{Mis fuentes principales son el table(ST\_MODG3, KMcluster) y el esquema de centros radial.}}\label{mis-fuentes-principales-son-el-tablest_modg3-kmcluster-y-el-esquema-de-centros-radial.}

\begin{enumerate}
\def\labelenumi{\arabic{enumi}.}
\setcounter{enumi}{5}
\item
  APLICACION DE ALGORITMOS PREDICTIVOS
\item
  ARBOL DE DECISIONES
\end{enumerate}

Usaremos el árbol de decisiones para este análisis predictivo debido a
la simplicidad de la presentación de datos en nuestro Dataset, que nos
permite hacer una rápida lectura y comparativa en torno a los resultados
que arroje el algoritmo tras aplicarse.

La mexcla de variables categóricas y numéricas hace más complicado usar
la metodología de regresiones lineales, lo que nos lleva a elegir este
método como el principal.

Se utilizará el paquete Rpart, que es con el que hemos aprendido en la
lección 3 y que parece encajar perfectamente con nuestro caso,
utilizando todas las variables que hemos usado en el analisis

library(caret) ST\_new\textless{}- ST\%\textgreater{}\%select(sex, age,
address,Pstatus, Medu, Fedu, Mjob,
Fjob,studytime,traveltime,failures,higher,internet, goout,
Dalc,Walc,health, absences,G1, G2, G3) ARBOL \textless{}- rpart(G3
\textasciitilde{} ., data = ST\_new, method = ``class'') PRIN
\textless{}- varImp(ARBOL) rownames(PRIN){[}order(PRIN\$Overall,
decreasing=TRUE){]}

printcp(ARBOL)

plotcp(ARBOL)

\subsection{Encontramos que G1 y G2 examen son predictores clave
seguidos por niveles de asistencia, consumo de alcohol y trabajos de los
padres.}\label{encontramos-que-g1-y-g2-examen-son-predictores-clave-seguidos-por-niveles-de-asistencia-consumo-de-alcohol-y-trabajos-de-los-padres.}

\section{\texorpdfstring{La lógica del árbol consiste en sólo utilizar
``attendance, Fjob, G1 y G2'' como variables basadas en la correlación y
la colinealidad entre algunas de las otras
variables.}{La lógica del árbol consiste en sólo utilizar attendance, Fjob, G1 y G2 como variables basadas en la correlación y la colinealidad entre algunas de las otras variables.}}\label{la-logica-del-arbol-consiste-en-solo-utilizar-attendance-fjob-g1-y-g2-como-variables-basadas-en-la-correlacion-y-la-colinealidad-entre-algunas-de-las-otras-variables.}

\begin{enumerate}
\def\labelenumi{\arabic{enumi}.}
\setcounter{enumi}{5}
\tightlist
\item
  CONCLUSIONES
\end{enumerate}

El hecho de que muchas variables estén correlacionadas es directamente
proporcional a la baja significancia o capacidad en las predicciones
finales, ya que no consiguen destacarse como variables unicas.

Respecto al análisis exploratorio según tecnicas de visualización o
cruce de datos y aplicando clustering, se obtienen algunos resultados
contradictorios, o quizás podría decirse que confirmatorios, como ocurre
con el caso de la relación de ausencias, Alcohol entre y en fin de
semana, o salidas,

Así como variables que influyen menos en las notas finales y tienen una
repercusión escasa

Aun queda mucho por aprender en este fascinante campo. Con más tiempo y
exhaustividad modelos supervisados y no supervisados podrían converger
con satisfacción y mejorar la calidad de la certidumbre de los datos


\end{document}
